\documentclass[aspectratio=169]{beamer}

\usetheme{epfl}
\usepackage{tikz}
\usetikzlibrary{positioning}

\begin{document}
	
	\title{A very long presentation title to test the borders of this template}
	\subtitle{with an additional subtitle}
	\author{First Lastname}
	\labname{EPFL-LABNAME}
	\date{\today}
	
	\logofilename{epfl/epfl_red.png}
	\titlepageimage{epfl/lacleman}
	\event{name event}
	\presentation{name presentation}
	\speaker{FIRST LASTNAME}
	
	\begin{frame}[plain]
		\maketitle
	\end{frame}
	
	\begin{frame}{A Frame Title}
		
		\begin{columns}
			\column{.5\textwidth}
			
			
			This is the frame content
			\begin{itemize}
				\item with some itemize objects
				\item and another one
			\end{itemize}
			
			Enumerates should work too
			\begin{enumerate}
				\item one
				\item two
			\end{enumerate}
		
			\column{.5\textwidth}
			
			A second column with other content, maybe more text, or something \emph{emphasized}. For more, highlight in \emphred{red} or \emphblue{blue}.
			
			
			
		\end{columns}
		
	\end{frame}

	\begin{frame}{Colors}
		\centering
		\begin{tikzpicture}[xscale=2.6, yscale=-2]
		\tikzstyle{colorbox} = [minimum width=1.8cm, minimum height=1.8cm, text width=1.5cm, font=\scriptsize]
		
		\node(rouge) at (0,0)[fill=rouge,text=white, colorbox]{rouge};
		\node(leman) at (0,1)[fill=leman,text=white, colorbox]{leman};
		\node(grosseille) at (0,2)[fill=grosseille,text=white, colorbox]{grosseille};
		\node(canard) at (0,3)[fill=canard,text=white, colorbox]{canard};
		\node(montrose) at (1,0)[fill=montrose,text=white, colorbox]{montrose};
		\node(perle) at (1,1)[fill=perle,text=white, colorbox]{perle};
		\node(vertedeau) at (1,2)[fill=vertedeau,text=white, colorbox]{vertedeau};
		\node(rose) at (1,3)[fill=rose,text=white, colorbox]{rose};
		\node(acier) at (2,0)[fill=acier,text=white, colorbox]{acier};
		\node(soufre) at (2,1)[fill=soufre,text=white, colorbox]{soufre};
		\node(carotte) at (2,2)[fill=carotte,text=white, colorbox]{carotte};
		\node(zinzolin) at (2,3)[fill=zinzolin,text=white, colorbox]{zinzolin};
		\node(chartreuse) at (3,0)[fill=chartreuse,text=white, colorbox]{chartreuse};
		\node(marron) at (3,1)[fill=marron,text=white, colorbox]{marron};
		\node(ardoise) at (3,2)[fill=ardoise,text=white, colorbox]{ardoise};
		\node(taupe) at (3,3)[fill=taupe,text=white, colorbox]{taupe};
		
		\end{tikzpicture}
		
		
\end{frame}

\end{document}